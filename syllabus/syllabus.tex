% (The MIT License)
%
% Copyright (c) 2023 Yegor Bugayenko
%
% Permission is hereby granted, free of charge, to any person obtaining a copy
% of this software and associated documentation files (the 'Software'), to deal
% in the Software without restriction, including without limitation the rights
% to use, copy, modify, merge, publish, distribute, sublicense, and/or sell
% copies of the Software, and to permit persons to whom the Software is
% furnished to do so, subject to the following conditions:
%
% The above copyright notice and this permission notice shall be included in all
% copies or substantial portions of the Software.
%
% THE SOFTWARE IS PROVIDED 'AS IS', WITHOUT WARRANTY OF ANY KIND, EXPRESS OR
% IMPLIED, INCLUDING BUT NOT LIMITED TO THE WARRANTIES OF MERCHANTABILITY,
% FITNESS FOR A PARTICULAR PURPOSE AND NONINFRINGEMENT. IN NO EVENT SHALL THE
% AUTHORS OR COPYRIGHT HOLDERS BE LIABLE FOR ANY CLAIM, DAMAGES OR OTHER
% LIABILITY, WHETHER IN AN ACTION OF CONTRACT, TORT OR OTHERWISE, ARISING FROM,
% OUT OF OR IN CONNECTION WITH THE SOFTWARE OR THE USE OR OTHER DEALINGS IN THE
% SOFTWARE.

\documentclass[nobrand,anonymous,nodate,nosecurity]{huawei}
\usepackage{enumerate}
\usepackage{multicol}
\usepackage{href-ul}
\usepackage{ffcode}
\newcommand\REG{$^{\tiny{\textsf{\textregistered}}}$}
\newcommand\TM{$^{\tiny{\textsf{TM}}}$}
\begin{document}

{\sffamily{\bfseries\Large Project Management Beyond Agile}\\
Series of lectures by \href{https://www.yegor256.com}{Yegor Bugayenko} to be presented
to students of \href{https://innopolis.university/en/}{Innopolis University} in 2023.
% and \href{https://www.youtube.com/playlist?list=PLaIsQH4uc08woJKRAA7mmjs9fU0jeKjjM}{video recorded}}

The entire set of slide decks is in \href{https://github.com/yegor256/pmba}{yegor256/pmba} GitHub repository.

\begin{abstract}
Today, Agile has emerged as a widely-used term among managers overseeing software development projects. Nonetheless, it's important to note that Agile is not a management framework per se, but rather a set of guiding principles intended for managers already utilizing an established framework, such as IBM's RUP\REG{} or Microsoft's MSF\REG{}. Furthermore, the PMBOK™ by PMI\REG{} posits that project management is a deterministic endeavor, regulated by stringent rules and even laws. This course seeks to form a connection between the traditionally dry formalism of project management and the progressive practices of Agile/XP.
\end{abstract}

% \section*{Introduction}

\textbf{What is the goal?}\\
The main aim of this course is to enable students to comprehend the core principles of project management as outlined by PMBOK\TM{}. It further encourages them to implement these principles in practical scenarios, particularly in commercial and open-source software development projects.

\textbf{Who is the teacher?}\\
Yegor is developing software for more than 30 years, being a hands-on programmer
(see his GitHub account: \href{https://github.com/yegor256}{@yegor256})
and a manager of other programmers. At the moment, he is a director
of an R\&D laboratory in Huawei. His recent conference talks are in
\href{https://www.youtube.com/channel/UCr9qCdqXLm2SU0BIs6d_68Q}{his YouTube channel}.
He also published a \href{https://www.yegor256.com/books.html}{few books}
and wrote a \href{https://www.yegor256.com/contents.html}{blog} about software engineering
and OOP.
He previously taught a few courses in
Innopolis University (Kazan, Russia)
and HSE University (Moscow, Russia),
for example,
\href{https://github.com/yegor256/ssd16}{SSD16 (2021)},
\href{https://github.com/yegor256/eqsp}{EQSP (2022)},
and
\href{https://github.com/yegor256/ppa}{PPA (2023)}
(all videos are available).

\textbf{Why this course?}\\
Agile, viewed as a software development philosophy, can be highly effective when applied by those well-versed in essential project management principles, such as scope, cost, and risk management. However, as Agile's popularity rises, there's an observed decline in the understanding of project management as a scientific discipline, a trend noted among both new graduates and experienced software engineers and managers. This course aims to enhance such understanding, minimizing the tedium typically associated with traditional management disciplines.

\textbf{What's the methodology?}\\
Each lecture engages in an analysis of several practical scenarios within a software development team. The aim is to discern both productive and unproductive situations. From these observations, conclusions are drawn that help students gain a clearer understanding and improved perspective of their own management decisions.

\newpage
\section*{Course Structure}

Prerequisites to the course (it is expected that a student knows this):

\begin{itemize}
\item How to write code
\item How to design software
\end{itemize}

After the course a student \emph{hopefully} will understand:

\begin{itemize}
\item How to draw a Gantt Chart and what for?
\item How to fire an under-performing team member?
\item How to create and maintain a Risk List?
\item How to identify risks in a project?
\item How to do quantitative and qualitative risk analysis?
\item How to report project status to a project sponsor?
\item How to estimate project costs?
\item How to calculate project budget?
\item How to avoid ``Gold Platting''?
\item How to decompose project scope into work packages?
\item How to measure performance of each team member?
\item How to optimize critical path using CPM?
\item How to work with a traceability matrix?
\item How to specify requirements unambiguously?
\item How to organize the work of a Change Control Board?
\item How to motivate programmers for higher productivity?
\item How to structure a software development contract?
\end{itemize}

\newpage
\section*{Lectures}

The following topics are discussed:

\newlist{lectures}{enumerate}{10}
\setlist[lectures]{label*=\arabic*.}
\begin{lectures}
\item Integration Management
    \begin{itemize}
    \item How to read PMBOK?
    \item How to identify and specify the problem?
    \item How to establish project rules?
    \item How to organize decision making process?
    \item How to embrace the chaos?
    \item How to be a good manager?
    \end{itemize}
\item Scope Management
    \begin{itemize}
    \item How to set Definition of Done (DoD)?
    \item How to decompose a project to tasks?
    \item How to avoid Gold Platting?
    \item How to blame the product not yourself?
    \item How to avoid micro-management?
    \end{itemize}
\item Time Management
    \begin{itemize}
    \item How to avoid Daily Stand-ups?
    \item How to stay away from Gantt Charts?
    \item How to avoid playing Planning Poker?
    \end{itemize}
\item Cost Management
    \begin{itemize}
    \item How to estimate project budget?
    \item How to pay what they deserve?
    \item How to pay 10x to 10x programmers?
    \item How to stop paying for your team education?
    \end{itemize}
\item Quality Management
    \begin{itemize}
    \item How to differentiate QA from testing?
    \item How to organize code reviews?
    \item How to get rid of altruism?
    \item How to aim for speed instead of quality?
    \end{itemize}
\item \emph{Human} Resource Management
    \begin{itemize}
    \item How to not spend two hours for an interview?
    \item How to motivate people?
    \item How to measure productivity of a dev team?
    \item How to measure productivity of a research team?
    \item How to boost team morale?
    \item How to fire painfully?
    \end{itemize}
\item Communications management
    \begin{itemize}
    \item How to avoid technical meetings?
    \item How to use Ticket Tracking Systems?
    \item How to avoid emails?
    \item How to work remotely?
    \item How to punish your team?
    \end{itemize}
\item Risk management
    \begin{itemize}
    \item How to build a risk list?
    \item How to prevent failures?
    \item How to respect and not trust your team?
    \end{itemize}
\item Procurement Management
    \begin{itemize}
    \item How to supervise an external team?
    \item How to avoid hourly pay?
    \item How to control their quality?
    \item How to measure productivity of an external team?
    \end{itemize}
\item Stakeholder Management
    \begin{itemize}
    \item How to be a good office slave (for your boss)?
    \item How to make your boss happy?
    \item How to be honest with a customer?
    \item How to bill incrementally?
    \item How to pass PMI exam?
    \end{itemize}
\end{lectures}

\newpage
\section*{Grading}

Students have to form groups of 2--4 people in each one.

Each group should pick a research topic from the list suggested by the teacher
at the first lecture. A group is allowed to pick its own topic, if approved
by the teacher at the first lecture.

Each group should do a research and write a research paper, according
to this \href{https://www.yegor256.com/2022/08/24/research-paper-template.html}{recommendation}.
The length of the pager may be no longer than four pages
\href{https://ctan.org/pkg/acmart}{acmart/sigplan} format (two columns).

The paper must be submitted to \href{http://www.icse-conferences.org/}{ICSE} or
\href{https://www.esec-fse.org/}{ESEC/FSE} conference
(student tracks and workshops are allowed).

The is no exam at the end of the course. Instead, the quality of the paper
will be evaluated by the teacher.

The attendance is tracked at the lectures. If 75\% of all lectures were attended,
it guarantees ``C'' (otherwise ignored).

\newpage
\section*{Learning Material}

The following books are highly recommended to read (in no particular order):

\begin{multicols}{2}\small\raggedright
{Rita Mulcahy}, \emph{PMI Exam: Preparation Course}, 2018\\[3pt]
{Yegor Bugayenko}, \emph{Code Ahead}, 2019\\[3pt]
Blog posts of Yegor Bugayenko, \href{https://www.yegor256.com/tag/management}{on his blog}\\[3pt]
\end{multicols}

\end{document}
